Interview mit: Lukas Adler
Qualifikation: 
Bachelor und Master Wirtschaftsinformatik an der Universität Mannheim.
Für Master Angular gelernt. Masterthesis bei CGI.
Hivi im Java Umfeld mit u.a. Spring u. Spring Boot, viel backend.
Dann Acando als Consultant.
Aktuell im front-end lead im Projekt mit Angular. Back-end mit Spring Boot.
Feature Entwicklung, Architektur Entwürfe, etc.

Criteria: 

License: 
directly applicable

Target Platforms: 
modifications necessary: 
- Browsers as target platform. Different browsers have certain specifics.
- For the target platforms like mobile, windows, mac etc. the browsers should be supported.

Supported development platforms: 
directly applicable

Distribution channels:
modifications: 
- deployment on web server, framework can have built in support for deployment
- less of a concern for web because there are no concise distribution channels like in mobile with google play and app store

Monetization: 
terminology update

Internationalisation: 
directly applicable

Long-term Feasibility: 
directly applicable

Development Environment:
directly applicable

Preparation Time: 
directly applicable

Scalability: 
directly applicable but more important for back-end than front-end

Development process fit: 
directly applicable but less of a concern because regular frameworks mostly fit all development processes

UI Design: 
modifications: 
- usually development takes place directly in the browser
- no WYSIWYG editor needed, but fast building inside the browser relevant

Testing: 
modifications: 
- testing in all relevant browsers important, especially with html/css that gets interpreted differently in various browsers
- external influences less of a concern
- one example might be that adds should pause while the user tabs into a different tab

Continuous Delivery: 
modifications: 
- CI platform can be connected to an artifactory
- container can be placed inside kubernetes

Configuration Management: 
directly applicable (except static websites)

Maintainability: 
directly applicable

Extensibility:
directly applicable

Integrating custom code: 
directly applicable but native code less of a concern for web.

Pace of development: 
directly applicable

Access to device-specific hardware:
not very relevant in web development. very restricted in accessing device specific hardware.

Access to platform-specific functionality: 
Less relevant in web. Access is dependent on the browser. Thr browser will determine the access to storage etc.
Access to the local file system will be through the browser.

Support for Connected Devices: 
Not very relevant in web development

Input device heterogeneity: 
Less input possibilities in web. mostly touch, mouse and keyboard.

Output device heterogeneity:
mostly differing screen sizes.

Application life cycle:
not very relevant in web. 
maybe if for example a website wants to register that the tab is currently not active, certain tasks might be paused. 
the browser does a lot of the work for us.

System integration: 
applicable to web. mostly http communication in web.

Security:
modifications:
differentiate between front-end and back-end. 
data validation, sql injections etc.
https very important. for seo also.

Robustness:
update terminology. 
naive option in web to show failure site.

Degree of mobility:
not really relevant for web.
maybe offline functionality if the web app is regulary used on the go.
but in general website work when there is an internet connection.

Look and feel:
not as relevant for web. There are certain guidelines for how to design for web, but the website doesnt aim to look different in safari or chrome etc.

Performance: 
modifications: 
content delivery network. website is spread across different servers so it is accessible from all over the world. 
battery drain not as much of a concern but network usage for example very important. 

Usage patterns:
modifications: 
website load times are short, website is übersichtlich, feedback for inputs. 

User authentication: 
modifications:
authentication via cookies or jwt. information from web server will only come back when the user is autheticated. 
web specific authentication: classic email and password, single sign on or two factor. key files, social login via facebook, google etc.


Missing:
- Search Engine Optimization
- Accessibility




 


