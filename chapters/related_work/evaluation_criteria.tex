\section{Evaluation framework for cross-platform app
development approaches}

Rieger and Majchrzak used a combination of literature review and expert consultation to determine which criteria were relevant for evaluating cross-platform app development frameworks. They conducted a systematic literature review of existing evaluation frameworks and studies, as well as a survey of experts in the field. The results of these efforts were used to develop an initial set of criteria, which were then refined through further expert consultation and feedback. The final set of criteria was then validated through case studies and expert evaluation.
This section provides an overview of the final criteria, along with concise explanations of each criterion.

\subsection{License} 
This criterion emphasizes the importance of considering the license of a cross-platform framework, particularly for commercial development. It acknowledges that license terms can have implications not only for the developed apps but also for the framework itself. It suggests evaluating the permissiveness of the license, the long-term feasibility of the framework, and the pricing model associated with it. Open-source frameworks are typically freely distributed under varying permissive regulations, while premium options may charge for maintenance and consultancy services.

\subsection{Target Platforms} 
This criterion highlights the significance of evaluating the supported target platforms of a cross-platform framework. With Android and iOS dominating the smartphone and tablet market, supporting these platforms is essential. However, widening cross-platform app development to other device categories can also increase the number of attractive platforms. It recognizes that different versions of a platform may introduce significant differences, and developing for them may be similar to developing for distinct platforms. The criterion acknowledges the importance of supporting novel features exploited by flagship devices to reach early adopters, as well as considering the update behavior and capabilities of different devices in various markets. It also mentions the importance of supporting combined apps that bridge multiple device classes, such as second screen apps or companion apps for smartphones and smartwatches.

\subsection{Development Platforms} 
This criterion focuses on the development platforms supported by a cross-platform framework. It acknowledges that developers may use different programming languages and development environments, and the framework should support interoperability and flexibility in expressing custom business logic and advanced configurations. It suggests evaluating the degree of flexibility in terms of hardware and software choices and the ability to integrate with additional app development tasks, such as UI and UX design.

\subsection{Distribution Channels} 
Distribution channels refer to the platform- or vendor-specific app stores such as the Apple App Store and Google Play where users commonly acquire new apps. Not all types of apps can be uploaded to all stores, so it's essential for a cross-platform framework to support a wide variety of relevant stores. Moreover, the framework compatibility with app store restrictions and submission regulations varies, and some well-integrated frameworks may offer additional features such as app rating and roll-out support for updates

\subsection{Monetization}
Monetisation pertains to the various ways apps can generate revenue. The potential monetisation models include paid apps, freemium apps (free with in-app purchases for additional features), paidmium apps (paid downloads with in-app purchases), in-app advertising, and free apps serving a specific business purpose. Frameworks may offer varying levels of support for these monetisation models, and good support may include features such as interfaces to payment providers, pre-designed functionality for in-app payments, support for various types of advertisements, and access to advertising networks.

\subsection{Internationalization}
This criterion relates to how well an app or platform can be distributed and used globally. It addresses the need for localization and internationalization of apps to cater to different languages and cultural nuances. This could involve translation capabilities, support for multiple languages, and conversion tools for localized content such as dates, currencies, and units. Internationalization also includes considerations for legal and geographic restrictions that may prevent the app from being distributed in certain regions.

\subsection{Long-term Feasibility}
This relates to the strategic decision in choosing an app development approach and the commitment over multiple years. The initial investment might be substantial and there could be a risk of technological lock-in, particularly for small companies. The long-term feasibility can be assessed through the framework's maturity, stability, and activity. This criterion also considers the reputation of stakeholders supporting the framework, the potential for technological advancements, and availability of commercial "premium support."

\subsection{Development Environment}
This involves the maturity and feature-richness of integrated development environments (IDEs) used for app development. A good IDE helps increase productivity with features like auto-completion, integrated library documentation, built-in debuggers, and emulators. If the framework does not enforce a specific IDE, developers can work in their accustomed workflows, reducing the initial setup effort.

\subsection{Preparation Time}
This criterion is about how quickly and effectively a developer can learn to use a new framework. It takes into account the required technology stack, the number and type of supported programming languages, and the familiarity of programming paradigms. A low entry barrier, comprehensive API documentation, guides, tutorials, and code examples make a framework more accessible and easier to learn.

\subsection{Scalability}
This criterion emphasizes on the ability of an app or platform to efficiently handle the growth in users and functionality. It involves proper modularization and structuring of the app into subcomponents and architectural considerations, which can make the development process more efficient and allow for more developers to be added as the app's functionality grows.

\subsection{Development Process Fit}
This criterion pertains to the compatibility of the development framework with different software development methodologies (such as Waterfall, Agile, etc.). It requires the framework to be adaptable to the custom ways of software development. This criterion also considers the effort required to create a minimum viable product and how well the framework supports scalability and modularity in relation to the development methodology.

\subsection{UI Design}
This relates to the importance of UI design in developing user-centered applications. The criterion considers the challenges posed by the varied device hardware and the need for flexible UI designs that adapt to different screen sizes. It also touches on the usefulness of WYSIWYG editors for designing interfaces for multiple devices and the need for support for platform-adequate designs.

\subsection{Testing}
This criterion focuses on the need for thorough testing of all components of apps, including UI, business logic, and more. It discusses the importance of considering mobile-specific scenarios and simulating external influences, providing meaningful error reporting and logging functionalities, and supporting remote debugging for more realistic test results.

\subsection{Continuous Delivery}
This criterion highlights the importance of life cycle support beyond testing, including deployment. A good framework will simplify deployment, potentially supporting the generation of native apps, signed packages, and even assisting with deploying these to devices or app stores. Continuous delivery platforms that automate building, testing, and deploying an app can be beneficial, especially for agile development. Advanced build options and continuous app store integration might also be offered by the framework.

\subsection{Configuration Management}
This criterion focuses on the ability of the framework to handle different configurations of the app, such as multiple roles (user and administrator), theming/branding variations, and regional peculiarities. It considers the framework's support for supplying different app packages for different versions or dynamically transitioning between versions without re-installing the app.

\subsection{Maintainability}
This criterion addresses the ease of maintaining the app codebase over time. While it can be challenging to quantify maintainability, factors such as code readability, use of design patterns, in-code documentation, and familiarity with the framework's programming paradigms and practices play a role. The reusability of source code across projects and the portability to other software projects are also considered.

\subsection{Extensibility}
This criterion evaluates the framework's ability to extend its functionality beyond the provided features. It assesses whether custom components can be added and third-party libraries can be included to meet project-specific requirements. This includes extensions for UI, access to device features, and libraries for common tasks.

\subsection{Custom Code Integration}
This criterion deals with the capability of integrating native code or third-party libraries into the app, even within a cross-platform development approach. It recognizes that there may be cases where certain functionalities cannot be achieved through extensions alone, and the ability to access platform-specific APIs or reuse existing native code becomes necessary.

\subsection{Pace of Development}
This criterion emphasizes the speed and efficiency of development with the framework. Factors such as the amount of boilerplate code required, availability of pre-defined functionality for common requirements, and overall development speed impact the variable costs and return on investment of the development process.

\subsection{Hardware Access}
This criterion focuses on the framework's ability to access and utilize the hardware features of mobile devices. It considers support for various sensors, cameras, GPS, accelerometers, and other device-specific capabilities that contribute to the versatility and functionality of the app.

\subsection{Platform Functionality}
This criterion evaluates the framework's support for accessing platform-specific functionalities such as file system access, database storage, network connection information, battery status, and in-app browser support. It also considers the ability to extend the app with complex business logic using general-purpose programming languages.

\subsection{Connected Devices}
This criterion examines the framework's support for interacting with and utilizing connected devices, including wearables, sensor/actuator networks, and other gadgets. It assesses the framework's ability to access data, sensors, and functionalities of connected devices and the provision of additional UI components if applicable

\subsection{Input Heterogeneity}
This criterion addresses the diverse range of input methods available on mobile devices, including keyboards, touch screens, hardware buttons, voice recognition, gestures, and more. It emphasizes that cross-platform frameworks should provide developers with access to these input methods, consider device-specific input limitations, and respect platform-specific input patterns.

\subsection{Output Heterogeneity}
This criterion focuses on the variability of output options offered by mobile devices, such as different screen sizes, resolutions, formats, color palettes, frame rates, and opacity levels. It acknowledges the challenges of adapting to device-specific output capabilities and the need to handle context changes, such as day/night screen mode.

\subsection{App Life Cycle}
This criterion highlights the importance of supporting the different stages of an app's life cycle, including starting, pausing, continuing, and exiting the app. It also considers multithreading, background services, and notifications that extend the app's states beyond the graphical user interface.

\subsection{System Integration}
This criterion focuses on the ability of the framework to integrate with backend systems, support data exchange protocols and formats, enable web service consumption, facilitate inter-app communication, and accommodate workflow-oriented use cases. It also emphasizes the need for customization options to align with corporate identities or design guidelines.

\subsection{Security}
This criterion addresses app security and emphasizes the framework's support for ensuring confidentiality, integrity, and control of app data. It includes considerations such as managing access permissions, securing sensitive data through encryption, using secure data transfer protocols, and preventing security vulnerabilities.

\subsection{Robustness}
This criterion emphasizes the need for apps to handle unsupported or restricted features gracefully and include fallback mechanisms. It also considers fault-tolerant and resilient mechanisms, such as handling denied permissions, deactivated sensors, and poor or unavailable internet access. Offline capabilities and data caching for synchronization are mentioned as examples.

\subsection{Degree of Mobility}
This criterion recognizes that different apps have varying degrees of mobility requirements, which go beyond infrastructure considerations. It categorizes apps into four levels of mobility: stationary, mobile, wearable, and autonomous. The criterion highlights that app mechanics and features need to adapt accordingly to the level of mobility, considering context information, personal preferences, and self-adaptation capabilities.

\subsection{Look and Feel}
This criterion emphasizes that the UI elements provided by a cross-platform framework should have a native look and feel, resembling the interface of the target platform rather than a web page. It suggests evaluating elements, views, and interaction possibilities based on the human interface guidelines provided by platform vendors. The criterion acknowledges that form-based interfaces are relatively simple to implement, while richer interfaces with 2D animations, 3D environments, and multimedia features can be more challenging for cross-platform frameworks.

\subsection{Performance}
This criterion highlights the importance of app performance for user acceptance. It includes aspects such as app load time, responsiveness to user interaction, perceived speed of network access, and stability. Objective performance measures such as start-up time, wake-up time after interruptions, shut-down time, CPU load, memory usage, battery drain, and download size can be evaluated. The criterion emphasizes the need for careful balancing of performance aspects, as a sole focus on performance may negatively impact other criteria, and optimization for other criteria may impact performance.

\subsection{Usage Patterns}
This criterion acknowledges that apps are used in typical patterns and should align with users' expectations and workflows. It includes considerations such as providing an "instant on" experience, preserving unsaved data after app closure or device reboot, maintaining data availability during temporary loss of connectivity, integrating with common apps for interaction and sharing, supporting data synchronization across multiple devices, and leveraging platform-wide services such as notification centers and document storage.

\subsection{User Authentication}
This criterion highlights the increasing importance of user management in apps, ranging from local single-user scenarios to cloud-based accounts with multi-device account management and role-based access rights. It addresses user authentication methods, session management, caching login information, and the need for frameworks to offer various authentication options, including traditional pins, passwords, gestures, biometric information, and voice recognition. The criterion emphasizes the significance of user authentication from the user perspective and its impact on app functionality and security.

