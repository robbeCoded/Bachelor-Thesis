
This section provides an overview of the final criteria and concise explanations for each.

\subsection{Infrastructure}
\textbf{(I1) License:}
This criterion emphasizes the significance of contemplating a cross-platform framework's license, especially for commercial development. It recognizes that license terms can have implications not only for the developed applications but also for the framework itself. It suggests evaluating the permissiveness of the license, the framework's long-term viability, and the associated pricing model. Open-source frameworks usually get distributed free of charge under a variety of permissive regulations, whereas premium alternatives may charge for maintenance and consulting services.

\textbf{(I2) Target Platforms:} 
This criterion emphasizes the importance of evaluating a cross-platform framework's supported target platforms. With Android and iOS dominating the smartphone and tablet markets, it is essential to support these platforms. Nevertheless, expanding cross-platform app development to other device categories can also increase the number of desirable platforms. It acknowledges that various versions of a platform may introduce substantial differences, and that developing for them may resemble developing for distinct platforms. The criterion recognizes the significance of supporting novel features utilized by flagship devices to reach early consumers, as well as taking into account the update behavior and capabilities of different devices in different markets. In addition, the significance of supporting combined apps that span multiple device classes, such as second-screen apps and companion apps for smartphones and smartwatches, is emphasized.

\textbf{(I3) Development Platforms:} 
This criterion focuses on the development platforms a cross-platform framework supports. It recognizes that developers may use a variety of programming languages and development environments, and the framework must support interoperability and flexibility in articulating custom business logic and advanced configurations. It suggests evaluating the hardware and software options as well as the ability to integrate with additional app development tasks, such as UI and UX design.

\textbf{(I4) Distribution Channels:} 
Distribution channels refer to the platform- or vendor-specific app stores, such as the Apple App Store and Google Play, from which consumers acquire new applications. It is essential for a cross-platform framework to support a wide range of relevant app stores, as not all app types can be uploaded to every store. Additionally, framework compatibility with app store restrictions and submission regulations varies, and some well-integrated frameworks may offer additional features such as app rating and update rollout support.

\textbf{(I5) Monetization:}
Monetization refers to the diverse methods by which applications can generate revenue. Paid apps, freemium apps (free with in-app purchases for additional features), paidmium apps (paid installations with in-app purchases), in-app advertising, and free apps serving a specific business purpose are potential monetization models. Frameworks may provide varying degrees of support for these monetisation models, with excellent support including interfaces to payment providers, pre-designed functionality for in-app payments, support for various types of advertisements, and access to advertising networks.

\textbf{(I6) Internationalization:}
This criterion relates to the global distribution and usability of an application or platform. It addresses the need for localization and internationalization of applications in order to accommodate linguistic and cultural differences. This may include translation capabilities, multiple language support, and conversion tools for localized content such as dates, currencies, and units. Legal and geographical restrictions that prohibit the app from being distributed in certain regions are also taken into account during internationalization.

\textbf{(I7) Long-term Feasibility:}
This concerns the strategic decision to select an app development methodology and the multi-year commitment. The initial investment may be substantial and technological lock-in may be a concern, especially for small businesses. The framework's maturation, stability, and activity can be utilized to assess its long-term viability. The criterion also takes into account the reputation of the stakeholders supporting the framework, the potential for technological advancements, and the availability of commercial "premium support."

\subsection{Development}
\textbf{(D1) Development Environment:}
This refers to the maturation and feature-richness of app development integrated development environments (IDEs). With features such as auto-completion, integrated library documentation, debuggers, and emulators, a decent IDE increases productivity. If the framework does not mandate a particular IDE, developers can work in their preferred procedures, thereby reducing the initial setup effort.

\textbf{(D2) Preparation Time:}
Preparation Time addresses how quickly and effectively a developer can pick up a new framework. It considers the necessary technology stack, the quantity and variety of supported programming languages, and the familiarity of programming paradigms. A low barrier to entry, comprehensive API documentation, guides, tutorials, and code examples make a framework more accessible and simpler to learn.

\textbf{(D3) Scalability}
Scalability emphasizes an app's or platform's capacity to effectively manage the growth of users and functionality. It involves the correct modularization and structuring of the app's subcomponents and architectural considerations, which can make the development process more efficient and enable the addition of more developers as the app's functionality expands.

\textbf{(D4) Development Process Fit:}
This criterion relates to the development framework's compatibility with various software development methodologies (such as Waterfall, Agile, etc.). It is necessary for the framework to be adaptable to specific software development techniques. This criterion also evaluates the effort required to produce a minimum viable product and the framework's support for scalability and modularity in relation to the development methodology.

\textbf{(D5) UI Design:}
This relates to the significance of UI design in user-centered application development. The criterion takes into account the difficulties posed by diverse device hardware and the need for UI designs that are adaptable to different screen sizes. It also discusses the utility of WYSIWYG editors for designing interfaces for multiple devices and the need for platform-appropriate designs to be supported.

\textbf{(D6) Testing:}
This criterion emphasizes the need for comprehensive testing of all app components, including UI, business logic, and others. For more realistic test results, the necessity of considering mobile-specific scenarios and simulating external influences, providing meaningful error reporting and logging capabilities, and facilitating remote debugging is addressed.

\textbf{(D7) Constant Delivery:}
Constant Delivery emphasizes the significance of life cycle support that extends beyond testing to deployment. A good framework will facilitate deployment by possibly supporting the generation of native applications, signed packages, and even assisting with their deployment to devices or app stores. Especially for agile development, continuous delivery platforms that automate the construction, testing, and deployment of an application can be advantageous. In addition to continuous app store integration, the framework may also offer advanced build options.

\textbf{(D8) Configuration Management:}
This criterion concentrates on the framework's ability to handle various configurations of the application, including multiple roles (user and administrator), theming/branding variations, and regional distinctions. It evaluates the framework's support for providing various app packages for different versions or dynamically switching between versions without reinstalling the app.

\textbf{(D9) Maintainability:}
Maintainability addresses the efficiency of codebase maintenance over time. Maintainability can be difficult to quantify, but code readability, design pattern usage, in-code documentation, and familiarity with the framework's programming paradigms and practices all play a role. Also considered are the reusability of source code across projects and the portability to other software projects.

\textbf{(D10) Extensibility:}
This evaluates the capability of the framework to extend its functionality beyond the provided features. It determines whether custom components and third-party libraries can be added to satisfy project-specific needs. This includes extensions for the user interface, access to device features, and common task libraries.

\textbf{(D11) Integrating custom code:}
Custom Code Integration relates to the app's capacity to incorporate native code or third-party libraries, even within a cross-platform development framework. It acknowledges that there may be instances where certain functionalities cannot be accomplished through extensions alone, necessitating access to platform-specific APIs or the ability to reuse existing native code.

\textbf{(D12) Pace of Development:}
This values the speed and effectiveness of development with the framework. The cost and return on investment of the development process are affected by factors like the amount of boilerplate code required, the availability of predefined functionality for common requirements, and the overall development pace.

\subsection{App}
\textbf{(A1) Access to Hardware:}
Access to Hardware emphasizes on the framework's access to and utilization of mobile devices' hardware features. It takes into account support for various sensors, cameras, GPS, accelerometers, and other device-specific capabilities that contribute to the app's versatility and functionality.

\textbf{(A2) Functionality of the Platform:}
This criterion evaluates the framework's support for gaining access to platform-specific features, such as file system access, database storage, network connection information, battery status, and in-app browser support. It also considers the capacity to add complex business logic to the application using general-purpose programming languages.

\textbf{(A3) Connected Devices:}
This evaluates the framework's support for interacting with and making use of connected devices, such as wearables, sensor/actuator networks, and other devices. It evaluates the framework's capacity to access the data, sensors, and capabilities of connected devices, as well as the availability of additional UI components, if pertinent.

\textbf{(A4) Input Heterogeneity:}
Heterogeneity addresses the wide variety of input methods available on mobile devices, such as keyboards, touch displays, hardware buttons, voice recognition, and gestures. It emphasizes that cross-platform frameworks should give developers access to these input methods, take device-specific input limitations into account, and respect platform-specific input patterns.

\textbf{(A5) Output Heterogeneity:}
Outcome Heterogeneity focuses on on the wide range of output options offered by mobile devices, including various screen sizes, resolutions, formats, color palettes, frame rates, and opacity levels. It recognizes the difficulties associated with adapting to device-specific output capabilities and the need to manage context changes, such as day/night screen mode.

\textbf{(A6) App Life Cycle:}
This criterion stresses the significance of supporting the various stages of an app's life cycle, such as launching, halting, continuing, and closing. Multithreading, background services, and notifications that extend the app's conditions beyond the graphical user interface are also considered.

\textbf{(A7) System Integration:}
System Intergration focuses on the framework's ability to integrate with backend systems, support data exchange protocols and formats, facilitate inter-app communication, and support workflow-oriented use cases. In addition, it accentuates the importance of customization options that align with corporate identities or design guidelines.

\textbf{(A8) Security:}
This addresses the framework's support for ensuring the confidentiality, integrity, and control of app data. It includes the management of access permissions, the encryption of sensitive data, the use of secure data transmission protocols, and the prevention of security vulnerabilities.

\textbf{(A9) Robustness:}
This criterion emphasizes that applications must effectively manage unsupported or restricted features and include fallback mechanisms. It also takes into account fault-tolerant and resilient mechanisms, such as resolving denied permissions, deactivated sensors, and poor or nonexistent internet access. Examples include offline capabilities and data caching for synchronization.

\textbf{(A10) Mobility Level:}
This acknowledges that various applications have varying degrees of mobility needs that extend beyond infrastructure considerations. It divides applications into four mobility categories: stationary, mobile, wearable, and autonomous. The criterion emphasizes that the app's functionality and features must adapt to the user's mobility level, taking into account context information, personal preferences, and self-adaptation capabilities.

\subsection{Usage Perspective}
\textbf{(U1) Look and Feel:}
This emphasizes that the user interface elements provided by a cross-platform framework should have a native appearance and feel. It suggests evaluating elements, views, and interaction options in accordance with platform vendors' human interface guidelines. The criterion recognizes that form-based interfaces tend to be easy to implement, whereas richer interfaces with 2D animations, 3D environments, and multimedia capabilities can be more difficult for cross-platform frameworks.

\textbf{(U2) Performance:}
Performance of the application is significant for user acceptance. It includes aspects such as app load time, user interaction responsiveness, perceived network access speed, and stability. Start-up time, wake-up time after interruptions, shutdown time, CPU load, memory usage, battery depletion, and download size are objective performance metrics that can be evaluated. The criterion emphasizes the importance of carefully harmonizing performance aspects, as a singular focus on performance may negatively affect other criteria, and optimizing for other criteria may affect performance.

\textbf{(U3) Usage Patterns:}
This criterion recognizes that applications are utilized in recurring patterns and should align with user expectations and workflows. It involves providing a "instant on" experience, preserving unsaved data after app termination or device reboot, maintaining data availability during temporary loss of connectivity, integrating with common apps for interaction and sharing, supporting data synchronization across multiple devices, and leveraging platform-wide services such as notification centers and document storage.

\textbf{(U4) User Authentication:}
The last criterion, User Authentication, emphasizes the growing significance of user management in mobile applications, from local single-user scenarios to cloud accounts with multi-device account management and role-based access privileges. It addresses user authentication methods, session management, preserving login information, and the requirement for frameworks to provide multiple authentication options, including traditional PINs, passwords, gestures, biometric information, and voice recognition. The criterion emphasizes the importance of user authentication from the user's perspective and its impact on the functionality and security of the application.