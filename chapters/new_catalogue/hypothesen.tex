\section{Hypotheses}
Three hypotheses guide this exploration:

\begin{itemize}
    \item Some criteria defined by Rieger and Majchrzak (2019) may not be as relevant or applicable to the evaluation of web development capabilities of a given cross-platform framework.
    \item Certain criteria defined by Rieger and Majchrzak (2019) may need to be modified or nuanced to better suit the evaluation of web development capabilities of a given cross-platform framework.
    \item There may be additional criteria not covered by Rieger and Majchrzak (2019) that are important for evaluating the web development capabilities of a given cross-platform framework.
\end{itemize}

To test these hypotheses, a series of expert interviews has been be conducted. Additionally, a comprehensive literature review from the previous chapter, which examines how web development frameworks have been evaluated in the past, will also inform the refinement process. The insights derived from these interviews and the literature review will inform the refinement of the evaluation framework, ensuring it is well-suited for assessing both app and web development capabilities of cross-platform frameworks. This chapter will detail the process and findings of these expert interviews, the insights from the literature review, and the subsequent refinement of the evaluation framework.

\section{Methodology}
The methodology for this chapter involved a two-pronged approach: conducting expert interviews and performing a comprehensive literature review. 

For the expert interviews, a systematic process was followed. Experts in the field of web and cross-platform development were selected based on their experience and knowledge. The selection aimed to ensure a diverse range of perspectives and insights into the evaluation of web development capabilities of cross-platform frameworks.

The experts were presented with the criteria catalogue developed by Rieger and Majchrzak (2019). Each criterion was discussed in detail with the experts, focusing on its relevance and applicability to web development. The experts were asked to consider whether each criterion was suitable as is, or if it required changes to its content or wording to better fit the context of web development. 

In addition to discussing the existing criteria, the experts were also asked to consider if there were any important aspects of web development evaluation missing from the current catalogue. This allowed for the identification of potential new criteria that could enhance the catalogue's comprehensiveness and applicability to web development.

Simultaneously, a comprehensive literature review was conducted. This review focused on existing literature on how web development frameworks have been evaluated in the past. The aim was to identify any additional criteria or considerations that might be relevant for the evaluation of web development capabilities of cross-platform frameworks.

The insights derived from the expert interviews and the literature review were then used to assess and refine the criteria catalogue. The goal was to ensure that the catalogue is well-suited for evaluating both app and web development capabilities of cross-platform frameworks.

\section{Criteria Evaluation and Revision}
\textbf{(I1) License:}
This criterion emphasizes the significance of contemplating a cross-platform framework's license, especially for commercial development. It recognizes that license terms can have implications not only for the developed applications but also for the framework itself. It suggests evaluating the permissiveness of the license, the framework's long-term viability, and the associated pricing model. Open-source frameworks usually get distributed free of charge under a variety of permissive regulations, whereas premium alternatives may charge for maintenance and consulting services.

\emph{Expert Input:} The experts agreed that the License criterion is relevant for web development. They found no need for changes to its content or wording, suggesting that it is suitable as is for evaluating web development capabilities of cross-platform frameworks.

\emph{Literature Review:} The literature review confirmed the relevance of the License criterion for web development. The review found that the license is a common feature in many comparisons and evaluations of web development frameworks.


\textbf{(I2) Target Platforms:} 
This criterion emphasizes the importance of evaluating a cross-platform framework's supported target platforms. With Android and iOS dominating the smartphone and tablet markets, it is essential to support these platforms. Nevertheless, expanding cross-platform app development to other device categories can also increase the number of desirable platforms. It acknowledges that various versions of a platform may introduce substantial differences, and that developing for them may resemble developing for distinct platforms. The criterion recognizes the significance of supporting novel features utilized by flagship devices to reach early consumers, as well as taking into account the update behavior and capabilities of different devices in different markets. In addition, the significance of supporting combined apps that span multiple device classes, such as second-screen apps and companion apps for smartphones and smartwatches, is emphasized.

\textbf{(I3) Development Platforms:} 
This criterion focuses on the development platforms a cross-platform framework supports. It recognizes that developers may use a variety of programming languages and development environments, and the framework must support interoperability and flexibility in articulating custom business logic and advanced configurations. It suggests evaluating the hardware and software options as well as the ability to integrate with additional app development tasks, such as UI and UX design.

\textbf{(I4) Distribution Channels:} 
Distribution channels refer to the platform- or vendor-specific app stores, such as the Apple App Store and Google Play, from which consumers acquire new applications. It is essential for a cross-platform framework to support a wide range of relevant app stores, as not all app types can be uploaded to every store. Additionally, framework compatibility with app store restrictions and submission regulations varies, and some well-integrated frameworks may offer additional features such as app rating and update rollout support.

\textbf{(I5) Monetization:}
Monetization refers to the diverse methods by which applications can generate revenue. Paid apps, freemium apps (free with in-app purchases for additional features), paidmium apps (paid installations with in-app purchases), in-app advertising, and free apps serving a specific business purpose are potential monetization models. Frameworks may provide varying degrees of support for these monetisation models, with excellent support including interfaces to payment providers, pre-designed functionality for in-app payments, support for various types of advertisements, and access to advertising networks.

\textbf{(I6) Internationalization:}
This criterion relates to the global distribution and usability of an application or platform. It addresses the need for localization and internationalization of applications in order to accommodate linguistic and cultural differences. This may include translation capabilities, multiple language support, and conversion tools for localized content such as dates, currencies, and units. Legal and geographical restrictions that prohibit the app from being distributed in certain regions are also taken into account during internationalization.

\textbf{(I7) Long-term Feasibility:}
This concerns the strategic decision to select an app development methodology and the multi-year commitment. The initial investment may be substantial and technological lock-in may be a concern, especially for small businesses. The framework's maturation, stability, and activity can be utilized to assess its long-term viability. The criterion also takes into account the reputation of the stakeholders supporting the framework, the potential for technological advancements, and the availability of commercial "premium support."